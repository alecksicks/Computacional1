\documentclass[12pt,a4paper]{article}
\usepackage[spanish]{babel}
\usepackage{graphicx}
\usepackage{amsmath}
\usepackage{amsfonts}
\usepackage{amssymb}
\usepackage{gensymb}
\usepackage[left=2cm,right=2cm,top=2cm,bottom=2cm]{geometry}
\renewcommand{\baselinestretch}{1.5}
\usepackage[utf8]{inputenc}
\begin{document}

\begin{titlepage}
	\centering
	\includegraphics[width=0.45\textwidth]{Unisonlogo}\par
    \vspace{1.3cm}
	\scshape \LARGE Universidad De Sonora \par
	\vspace{1cm}
	\scshape \Large Licenciatura en Física \par
    \scshape \Large Física Computacional I \par
	\vspace{1.5cm}
	\huge \scshape Analísis de Mareas \par
	\vspace{2cm}
	\Large Alexis Andrey Martínez Padilla\par
    \Large Profesor Carlos Lízarraga Celaya\par
    \vspace{2.5cm}
    \Large 19 de Mayo del 2017
\end{titlepage}

\section{Breve Resumen}

Como continuación a la actividad anterior, ahora analizaremos la marea de Fort Point utilizando la transformada discreta de Fourier. Con ella llegaremos a algunos datos relevantes para el análisis de la amplitud, frequencia y periodo.

\section{Introducción}

\newpage

\section{Resultados Obtenidos}

\begin{center}
\includegraphics[scale=0.50]{fp1}
\end{center}

\begin{table}[]
\centering
\label{my-label}
\begin{tabular}{|c|c|c|c|}
\hline
Armónico & Periodo (hr)  & Frequencia (1/hr) & Amplitud (m)   \\ \hline
Primero  & 126.101694915 & 0.00793010752688  & 0.226192493208 \\ \hline
Segundo  & 124.0         & 0.00806451612903  & 0.6325385831   \\ \hline
Tercero  & 120.0         & 0.00833333333333  & 0.127062666561 \\ \hline
\end{tabular}
\end{table}

Y en la gráfica nos muestran los puntos dados:

\begin{center}[b!]
\includegraphics[scale=0.50]{fp2}
\end{center}

\newpage

\section{Código empleado}

\begin{verbatim}
#Se importaron las librearías necesarias
import pandas as pd
import numpy as np
import matplotlib.pyplot as plt
from pylab import *
import scipy
from scipy.fftpack import fft,fftfreq, fftshift

#Se leyó el archivo separado en comas y se le asignaron nombre a las columnas
df=pd.read_csv("FortPoint.csv",parse_dates=['Date Time'],header=int(0))
df.columns=['Date Time','Water Level','Sigma','O','F','R','L','Quality']

#Se utilizará solo el valor absoluto del nivel del agua, se observa el final del archivo para ver cuantos datos contiene
df1=df.loc[abs(df['Water Level']) < 0.09]
df.tail()

#Ahora se aplica la función de la transformada de Fourier
N_d = 7440
T = 1.0
y = df["Water Level"] 
yf = fft(y)
xf = fftfreq(N_d, T)
xf = fftshift(xf)
yplot = fftshift(yf)

graf = plt.plot(xf, 2.0/N_d *abs(yplot))
plt.xlim(-0.01,0.1)
plt.grid(True)

plt.xlabel('Frecuencia [1/hora]')
plt.ylabel('Amplitud [m]')
plt.title('Fort Point, New Hampshire')

fig = plt.gcf()
fig.set_size_inches(10, 5)
plt.show()

#Se buscan algunos máximos
a= np.absolute(yf/7440)
a[a[:,] > 0.05]

print(np.where(a[:,]>0.1))
b= a[a[:,]>0.05]
b

#Resulta ser que hay tres valores cercanos (59, 60 y 62), así que se usarán esos.
#Ahora buscaremos la amplitud, frecuencia y periodo de esos puntos
print( 'Primer Armónico notorio')
print('Amplitud=',np.absolute(yf[59,]/7440))
print('frecuencia=', xf[int(3720+59),])
print('periodo=', 1/xf[int(3720+59),])

print('Segundo Armónico notorio')
print('Amplitud=',np.absolute(yf[60,]/7440))
print('frecuencia=', xf[int(3720+60),])
print('periodo=', 1/xf[int(3720+60),])

print('Tercer Armónico notorio')
print('Amplitud=',np.absolute(yf[62,]/7440))
print('frecuencia=', xf[int(3720+62),])
print('periodo=', 1/xf[int(3720+62),])

#Por último se configura la gráfica y se muestra
fig = plt.gcf()
fig.set_size_inches(10, 5)


graf = plt.plot(xf, 1.0/7440 *abs(yplot))
plt.xlim(-0,0.01)
plt.grid(True)

plt.xlabel('Frecuencia [1/hora]')
plt.ylabel('Amplitud [m]')
plt.title('Fort Point, New Hampshire', fontsize=15)

plt.text(0.00793010752688,0.226192493208, '$N_1$')
plt.text(0.00806451612903,0.6325385831, '$N_2$')
plt.text(0.00833333333333,0.127062666561, '$N_3$')


plt.show()
\end{verbatim}

\section{Bibliografía}

\begin{enumerate}

\item  Carlos Lizárraga Celaya; Universidad de Sonora, Departamente de Física. \\ http://computacional1.pbworks.com/w/page/115478932/Actividad\%204\%20(2017-1)

\item https://en.wikipedia.org/wiki/Discrete-time\_Fourier\_transform

\item https://en.wikipedia.org/wiki/Fourier\_analysis

\item https://docs.scipy.org/doc/scipy-0.18.1/reference/tutorial/fftpack.html

\end{enumerate}

\end{document}

