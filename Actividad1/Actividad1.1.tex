\documentclass{article}

\usepackage[spanish]{babel}
\usepackage{graphicx}
\usepackage[utf8]{inputenc}
\begin{document}

\begin{titlepage}
	\centering
	\includegraphics[width=0.45\textwidth]{Unisonlogo}\par
    \vspace{1.3cm}
	\scshape \LARGE Universidad De Sonora \par
	\vspace{1cm}
	\scshape \Large Departamento de Física\par
    \scshape \Large Licenciatura en Física \par
	\vspace{1.5cm}
	\huge \scshape La Estructura de la Atmósfera\par
	\vspace{2cm}
	\Large Alexis Andrey Martínez Padilla\par
    \Large Profesor Carlos Lízarraga Celaya\par
    \vspace{2.5cm}
    \Large 29 de Enero del 2017
\end{titlepage}

\section{Breve Resumen}
    En este trabajo encontraremos una pequeña definición de en lo que consiste nuestra atmósfera y cada una de sus secciónes, muy brévemente resumido.

\section{Introducción}
    La atmósfera es una combinación de varios gases que permiten la vida en nuestro planeta. Está compuesta principálmenmte de Nitrógeno y Oxígeno (en forma molecular), y posee otros elementos pero en mucha menos distribución, entre ellos se encuentrán Hidrógeno, Argón, Helio, Neón, etc.
    
\section{Las Capas Atmosféricas}
    Cada capa se distingue de las demás por sus propiedades, a medida que uno va hacia arriba algunas características no actúan linealmente como la temperatura. A continuación se verán muy brevemente en que consiste cada una.

\begin{figure}[ht!]
\centering
\includegraphics[width=90mm]{atmo1}
\caption{Imágen por Comet Program \label{overflow}}
\end{figure}

\subsection{La Troposfera}
Esta es la capa donde vivimos, abarca hasta 15 kilómetros de Altitud y es donde todos los cambios climáticos ocurren. La temperatura en esta capa generalmente disminuye conforme aumenta la altitud. \\
Al final de esta capa existe la Tropopausa, que es donde los aviones tienen a realizar sus vuelos por su calma. Esta región varia siendo mas alta en lugares cálidos y mas baja en lugares fríos.

\begin{figure}[ht!]
\centering
\includegraphics[width=65mm]{tropo1}
\caption{Imágen CobaesGeo \label{overflow}}
\end{figure}

\subsection{La Estratosfera}
En la siguiente capa la temperatura aumenta con la altitud; esto es porque a lo mas alto se encuentra la capa de ozono, que es la que nos protege de los rayos UV del Sol, lo que nos dice que guarda calor. Abarca desde los 15 kiloómetros hasta los 50 kilómetros.

\begin{figure}[ht!]
\centering
\includegraphics[width=50mm]{stra1}
\caption{Imágen CobaesGeo \label{overflow}}
\end{figure}

\subsection{La Mesósfera}
Esta capa es algo similar a la troposfera, contiene un radio de oxigeno y nitrógeno semejantes y la temperatura disminuye con la altitud. Empieza a los 50 Kilómetros hasta los 80 kilómetros.

\begin{figure}[ht!]
\centering
\includegraphics[width=70mm]{meso1}
\caption{Imágen por EcologiaHoy \label{overflow}}
\end{figure}

\subsection{La Termósfera}
Esta es la ultima capa, esta por encime de los 80 kilómetros y su temperatura aumenta con su altitud. Esto siendo así puesto a que es calentada directamente por el Sol.

\begin{figure}[ht!]
\centering
\includegraphics[width=70mm]{termo1}
\caption{Imágen por EcuRed \label{overflow}}
\end{figure}

\section{Bibliografía}
AstroMia. (2013). La capa de aire que rodea la Tierra. 2017, de Astromia.com Sitio web: http://www.astromia.com/tierraluna/atmosfera.htm \\
\\
NCSU. (2013). Structure of the Atmosphere. 2017, de NC State University Sitio web: http://climate.ncsu.edu/edu/k12/.AtmStructure \\
\\
Ambientum. (1999). COMPOSICIÓN DE LA ATMÓSFERA. 2017, de Ambientum Sitio web: http://www.ambientum.com/enciclopedia_medioambiental/atmosfera/Composicion-de-la-atmosfera.asp \\
\\

\end{document}
